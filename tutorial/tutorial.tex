\documentclass[vipdfmx,a4paper,11pt]{jsarticle}
\usepackage{hyperref} % ハイパーリンク差し込み用
\usepackage{pxjahyper} % 目次の文字化け対策

\usepackage[dvipdfmx]{graphicx,xcolor}% ドライバ指定のため

%オートマトン
\usepackage{tikz}
\usetikzlibrary{automata,positioning}

% 数式
\usepackage{amsmath,amsfonts}
\usepackage{amssymb}
\usepackage{bm}
\usepackage{braket}
\usepackage{multicol}

% 画像
\usepackage[dvipdfmx]{graphicx}

% SI単位
\usepackage{siunitx}

%プログラムリスト用
\usepackage{listings,jvlisting}
\lstset{
basicstyle={\ttfamily},
identifierstyle={\small},
commentstyle={\smallitshape},
keywordstyle={\small\bfseries},
ndkeywordstyle={\small},
stringstyle={\small\ttfamily},
frame={tb},
breaklines=true,
columns=[l]{fullflexible},
numbers=left,
xrightmargin=0zw,
xleftmargin=3zw,
numberstyle={\scriptsize},
stepnumber=1,
numbersep=1zw,
lineskip=-0.5ex
}

\renewcommand{\thesubsection}{\arabic{section}-\arabic{subsection}}
\renewcommand{\thesubsubsection}{\arabic{section}-\arabic{subsection}.\arabic{subsubsection}}
\makeatletter
  \renewcommand{\theequation}{\arabic{subsection}.\arabic{equation}}
  \@addtoreset{equation}{subsection}
\makeatother
\makeatletter
% sectionの下マージンを小さく
\renewcommand{\section}{%
  \@startsection{section}{1}{\z@}%
  {0.4\Cvs}{0.1\Cvs}%
  {\normalfont\large\headfont\raggedright}}
\renewcommand{\subsection}{%
  \@startsection{subsection}{1}{\z@}%
  {0.4\Cvs}{0.1\Cvs}%
  {\normalfont\large\headfont\raggedright}}
\renewcommand{\subsubsection}{%
  \@startsection{subsubsection}{1}{\z@}%
  {0.4\Cvs}{0.1\Cvs}%
  {\normalfont\large\headfont\raggedright}}
\makeatother

\begin{document}

\title{CNSS tutorial(jp)}
\author{Takanori Saiki}
\date{\today}
\maketitle

\newpage

\section{はじめに}
CNSS(Cell Network Shape Simulation)は分子通信のシミュレーションをおこなう汎用シミュレータです。
シミュレーションにおいて基盤となるプログラムは用意されているので、ユーザはパラメータを変更することで、
あるいは独自に拡張をおこなうことで自身の考えたシミュレーションモデルを簡単に実装することができます。

\section{導入}
ここでは、CNSSを導入するための手順を説明します。使用環境はMac OSかLinuxを想定していますが、
いくつかのツールを揃えることでWindowsでも実行可能となります。

\subsection{ダウンロード}
GitHubのCNSSのページ(\href{https://github.com/saikiRA1011/CellNetworkShapeSimulation}{https://github.com/saikiRA1011/CellNetworkShapeSimulation})
からソフトウェアをダウンロードします。ダウンロード先はどこでも良いですが、シミュレーションの実行後に画像ファイルや
動画ファイルが作成されるため、容量に余裕のあるドライブに配置しましょう。

\subsection{セッティング}
CNSSではいくつかの外部ツールをシミュレーションの作成、実行に用いています。以下はそのツールとバージョンの一例です。
\begin{itemize}
  \item gcc 11.3.0 (clangでも良い)
  \item clang 13.1.6 (gccでも良い)
  \item Python 3.9.12
  \item pip 22.1.2
  \item make 3.81
\end{itemize}

Pythonで用いるライブラリは以下のコマンドから一括でインストールできます。

\begin{lstlisting}
  pip install -r requirements.txt
\end{lstlisting}



\end{document}